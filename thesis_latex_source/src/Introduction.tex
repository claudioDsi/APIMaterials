% !TEX TS-program = pdflatex


\section{Summary}
% Context
Leveraging the time-honored principles of modularity and reuse, modern software 
systems development typically entails the use of external libraries.
Rather than implementing new systems from scratch, developers look for, and try 
to integrate into their projects, libraries that provide functionalities of 
interest. Libraries expose their functionality through Application Programming 
Interfaces (APIs) which govern the interaction between a client project and the 
libraries it uses.

% Learning APIs
Developers therefore often face the need to learn new APIs.
The knowledge needed to manipulate an API can be extracted from various 
sources:~the API source code itself, the official website and documentation, 
Q\&A websites such as StackOverflow, forums and mailing lists, bug trackers, 
other projects using the same API, etc.
However, an official documentation often merely reports the API description 
without providing non-trivial example usages. Besides, querying informal 
sources such as StackOverflow might become time-consuming and 
error-prone~\cite{robillard2009makes}.
Also, API documentation may be ambiguous, incomplete, or 
erroneous~\cite{uddin2015api}, while API examples found on Q\&A websites may be 
of poor quality~\cite{nasehi2012makes}.

% APIs, mining, existing tools, etc.
Over the past decade, the problem of API learning has garnered considerable 
interest from the research community.
Several techniques have been developed to automate the extraction of API 
\emph{usage patterns}~\cite{Robillard:2013:AAP:2498733.2498776} in order to 
reduce developers' burden when manually searching these sources and to provide 
them with high-quality code examples. However, these techniques, based on 
clustering~\cite{Niu2017API, Wang2013Mining, Zhong2009MAPO} or predictive 
modeling~\cite{Fowkes:2016:PPA:2950290.2950319}, still suffer from high 
redundancy~\cite{Fowkes:2016:PPA:2950290.2950319} and---as we show later in the 
thesis---poor run-time performance.

To cope with these limitations, a new approach is proposed in this thesis to 
recommend to developers items that have been bought by similar users in similar 
contexts. Informally, the question the proposed system can answer is:

\begin{quote}
	\textit{``Which API methods should this piece of client code invoke, 
	considering that it has already invoked these other API methods?"}
\end{quote}
%We transpose this idea in the context of API recommendation:~should the method 
%currently being written (a customer) invoke (buy) a method from an API (a 
%product) given the context of the current project?

%%%%%%%%%%%%%%%%%%%%%%

A real big issue is how to perform a good enough recommendation in this 
context, balancing possible bias and putting the proper hints for the 
developer. Moreover, the form of the recommendation is also important because, 
in general, there are variety of possible suggestion such as code snippet, 
patterns for the methods, enhance documentation and all things that make a 
recommendation really usable for the current project. 

This work is developed within the European H2020 CROSSMINER 
project~\cite{CROSSMINER} that aims at conceiving techniques 
and tools for developing new software systems by reusing existing open source 
components. Nowadays, the complex software system are really big and it is not 
so easy to select and deploy a component in the right way. The planned 
CROSSMINER technical offering consists of the following analysis tools:
\begin{itemize}
	\item Source code analysis tools to extract and store knowledge from the 
	source code of a collection of open-source projects;
	\item Natural language analysis tools to extract quality metrics related to 
	the communication channels, and bug tracking systems of OSS projects by 
	using Natural Language Processing and text mining techniques;
	\item System configuration analysis tools to collect and analyse system 
	configuration artefacts;
	\item Workflow-based knowledge extractors that simplify the analysis of a 
	complex software system;
	\item Cross-project relationship analysis tools to manage a wider range of 
	open source project relationships, such as dependencies and conflicts, 
	based on user-defined similarity measures and the creation of project 
	clusters;
	\item Advanced integrated development environments that will allow 
	developers to adopt the  knowledge base and analysis tools directly from 
	the development environment, that providing alerts, recommendations, and 
	user feedback which will help developers to improve their productivity.
\end{itemize}
Figure \ref{fig:crossminerApproach} shows an overview of the CROSSMINER 
approach at work. In such a context, the work presented in this thesis wants to 
propose a novel tool that perform API function call recommendations in the 
context of Java projects. It is integrated in the CROSSMINER knowledge base 
component in a  flexible way. 



\begin{figure}[!t]
\includegraphics[width=14cm,height=14cm,keepaspectratio]{images/crossminer.png}
\centering
\caption{The CROSSMINER project at work}
\label{fig:crossminerApproach}
\end{figure}


\section{Research Objectives}
\begin{figure}[!h]
	\includegraphics[width=12cm,height=12cm,keepaspectratio]{images/Kb.png}
	\centering
	\caption{An overview of the CROSSMINER knowledge base}
	\label{fig:crossminerKB}
\end{figure}

Figure \ref{fig:crossminerKB} shows an overview of the CROSSMINER 
knowledge base underpinning the whole recommendation mechanism; the proposed 
approach gives support for the \code{APIrecommender} subcomponent in the 
picture. In particular, we combine the concepts of code cloning and patterns to 
retrieve real code snippets that show a concrete usage of the libraries used by 
the developer. We choose this approach because code snippets represent 
immediate hints in the developing context, as a concrete usage of a method or 
class is more relevant rather than a JavaDocumentation description or the list 
of imports. We exploit also the code cloning technique in order to search and 
retrieve possible suggestion.

\section{Thesis Structure}

The thesis is organized in the following chapters:

\begin{itemize}
	\item Chapter 2 introduces the context of this thesis. The notions of 
	recommendation systems and API mining are introduced. Code cloning 
	techniques are also overviewed as underpinning the API recommendation 
	approach proposed in Chapter 4 of this thesis.
		
%	In particular provides a general overview on the key concepts used in the 
%	developed of the proposed tool. These concepts are the definition of API 
%	and recommendation; there are very general terms and the literature are 
%	plenty of examples and definitions in such a way that it is impossible to 
%	give an exhaustive overview. For this reason, we try to define in the right 
%	manner the context suitable for the proposed approach, by avoiding the 
%	perfect coverage of the topic and limit ourself to a few but relevant 
%	concepts for our aims. Moreover, we provide a general state of the art of 
%	the code cloning domain, with some definitions as cloned fragment, sources 
%	units and comparison algorithms. We also present different code cloner to 
%	have an overview on the different approaches and finally, we present 
%	Simian, the code cloner used in the proposed approach.
	
	\item Chapter 3 presents an overview of existing approaches able to mine 
	APIs. A comparative table is presented at the end of the chapter with 
	respect to peculiar features.
	
%	Section 3 gives an overall view of the problem, considering the most 
%	used and famous tool used for facing the API recommendation problems. They 
%	use very different techniques and a state of the art is necessary to better 
%	understand approaches, level of the recommendations and possible issues 
%	that arise when we develop these kind of systems.The general API 
%	recommendations procedure involves the visit of the AST, some clustering 
%	techniques and a ranking as postprocessing phase and we compare various 
%	existing approaches. Among these works, we select CLAMS for its results to 
%	perform our recommendations and PAM  for validate our approach. 
	
	\item Chapter 4 presents the proposed approach, which relies on the combined
	adoption of the existing CLAMS and Simian tools presented in Chapter 3. In 
	particular, CLAMs is used to create snippets of code representing recurring 
	patterns in the analyzed APIs. Subsequently, Simian is used to analyze the 
	developer's context, that contains what she is developing and, in 
	particular, the fragment of code on which she would like to get 
	recommendations. As final outcome, the proposed approach is able to provide 
	developers with novel patterns in the form of snippets of code that 
	contains method invocation and all the variables that are needed to execute 
	them in the proper way.
	
	\item Chapter 5 presents an evaluation of the proposed recommendation 
	approach.
	%In the Section 5, we propose an evaluation framework based on AST of  the 
	%code.The aim of this framework is to validate in an empirical way the 	
	%produced recommendations by analysing the method declarations and 	
	%invocations. To do this, we use JavaParser to retrieve the snippet of 
	%code 	
	%that represent the context and Rascal, a meta programming language, to 	
	%parse the AST of the developer's project in order to represent in the 	
	%proper way the context, very important to evaluate if the suggested 	
	%patterns are the correct ones for it. 
	The evaluation is performed by 
	considering four metrics: \textit{precision}, \textit{recall}, 
	\textit{success rate} and \textit{F-measure}. 
%	All the metrics are explained in the related section and it is used on the 
%	list of method invocations retrieved by Rascal. Moreover, we use PAM as to 
%	compare the final results. 

	\item Chapter 6 concludes the thesis and performs an analysis of possible 
	future works.
%	 in this area, starting from an improvement in the 
%	recommendation format and alternative techniques with respect to the code 
%	cloning used for this work.
	
\end{itemize}


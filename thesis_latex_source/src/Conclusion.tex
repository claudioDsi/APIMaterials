


Aiming to assist developers in their development activities by mining open source software repositories, we propose a framework for providing API function calls. This topic has attracted attention from the research community. Our work is built based on some existing tools and it can be integrated into Eclipse. On the one side, we have Simian, a code cloner that is able to retrieve cloned code among files without needing the pre-processing or post-processing phases. On the other hand, we have CLAMS, a tool that gives as output a ranked list of patterns. The final goal is to provide API function call recommendations in the context of software complex system. This means that we have to analyze Java user's file put within a more bigger structure (as Eclipse project for example). So, a code cloner tool is not enough to generate proper recommendations at the level of code snippet and we integrate the concept of patterns thanks to CLAMS approach. From this, we reuse output files that contain useful patterns for our purposes. In this way, we provide recommendations at level of code snippet, which is closer to code and helps developers complete.%or to have a possible different implementation of the feature that he is developing. Notice that the provided recommendations are always related to a certain library at once. 

%Our approach leaves some open issues. %, those are strongly related with the API recommendations.

From the evaluation framework, it can be seen that the approach is strongly related to the developer's code snippet that Simian takes as input. From the comparison with PAM, we notice that the proposed approach can produce accurate recommendations. Only in the case of some values of recall and f-measure are inferior to those of than PAM. Another issue is that CLAMS patterns have a structure composed first by a list of variable used by the method calls and then the chain of method declaration to realize a specific feature. This structure may introduce some bias because Simian can find similarities only among the list of variables and not in the method calls. Moreover, Simian divides source code to analyze into blocks, limited by the threshold parameter. In some cases this can lead to false positives since Simian cannot identify a certain pattern in the code if it is not smaller than the threshold value. For example, in source code there are two methods that belong to a CLAMS pattern which includes another method invocations, if in the original source code the third invocation differs from that, Simian is not able to retrieve the pattern. Some threats could arise from the evaluation framework in which we used Rascal. In order to validate the approach, we had set up an Eclipse structure as shown in the related section. Although the structure is correct, it is manually built so it may affect the metrics, in particular precision and recall.
 
%From the point of view of improvements, 

We opt for a code cloner, however there are a lot of alternative approaches that can be exploited, like another code cloner or change completely the approach. We can beautify the output results that are presented in a simple file by setting up an Eclipse plugin, with all necessary components. Regarding the evaluation, we can provide a human survey by involving developers with different skills and knowledge to asses the provided results with a different perspective. Considering other approaches, many studies use probabilistic techniques, for example, define models or probability distribution. The central point of our work is the use of a code cloner to detect similarities between the pattern retrieved by CLAMS and the actual code snippet. Although Simian doesn't have any pre- or post-processing phases, we overcome this limitations by manually selecting source files and ranking the results, considering the duplicated lines of code. Moreover, we added also the AST concept in the evaluation framework. In this way, we are able to equip Simian with a new functionality that considers only the textual similarity. %Following this road, a possible improvement should be use a code cloner based on different similarity comparison (like suffix tree), but the challenge of integrate it in the Eclipse platform still remain opened. Most of code cloner are developed for simply analyze the developer's code and are used only to find clones in different files.   

%For future work, we are going to thoroughly evaluate our proposed approach by using other similar techniques as baseline, with the consideration of more data.

%. In this work we presented our proposed approach to deal with API recommendations
%As the precision of the mined code snippets (like false positive in the Simian analysis or in the CLAMS patterns). 